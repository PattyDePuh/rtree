%%%%%%%%%%%%%%%%%%%%%%% file rtree.tex %%%%%%%%%%%%%%%%%%%%%%
% Erweiterungen des R-Baums für räumliche Datenbankanfragen %
%%%%%%%%%%%%%%%%%%%%%%%%%%%%%%%%%%%%%%%%%%%%%%%%%%%%%%%%%%%%%


\documentclass[runningheads,a4paper]{llncs}

\usepackage{amssymb}
\setcounter{tocdepth}{3}
\usepackage{graphicx}
\graphicspath{ {./img/} }           % change graphics include path to ./img/

\usepackage[english, ngerman]{babel}
\usepackage[utf8]{inputenc}			    % for working umlaute
\usepackage[T1]{fontenc}            % wichtig für Trennung von Wörtern mit Umlauten
\usepackage{microtype}						  % verbesserter Randausgleich
\usepackage{pdfpages}

\usepackage{quoting}
\usepackage[
	babel,
	german=quotes,
	threshold=2,										% two lines ore more triggers a blockquote
	thresholdtype=lines
]{csquotes}
% \SetBlockEnvironment{quoting}			% real block quotes
\renewcommand\mkblockquote[4]{\leavevmode\llap{„}#1#2#3“#4}			% block quotes with german quotation marks

% quellen mangement
\usepackage[
	backend=biber,
	natbib=true,								% damit \citep{} etc. verwendet werden können
	style=authoryear-icomp,    	% Zitierstil
	isbn=false,                	% ISBN nicht anzeigen, gleiches geht mit nahezu allen anderen Feldern
	pagetracker=true,          	% ebd. bei wiederholten Angaben (false=ausgeschaltet, page=Seite, spread=Doppelseite, true=automatisch)
	maxbibnames=10,            	% maximale Namen, die im Literaturverzeichnis angezeigt werden
	maxcitenames=2,            	% maximale Namen, die im Text angezeigt werden, ab 2 wird u.a. nach den ersten Autor angezeigt
	autocite=inline,           	% regelt Aussehen für \autocite (inline=\parancite)
	block=space,               	% kleiner horizontaler Platz zwischen den Feldern
	backref=true,              	% Seiten anzeigen, auf denen die Referenz vorkommt
	backrefstyle=three+,       	% fasst Seiten zusammen, z.B. S. 2f, 6ff, 7-10
	date=short									% Datumsformat
]{biblatex}
\setlength{\bibitemsep}{0.6em}     	% Abstand zwischen den Literaturangaben
\setlength{\bibhang}{2em}        		% Einzug nach jeweils erster Zeile
% \renewcommand{\postnotedelim}{\addcolon\addspace}			% Doppelpunkt statt Komma vor der Seitenangabe in der Zitierung
% \DeclareFieldFormat{postnote}{#1}											% Kein einleitendes «S.» vor der Seitenangabe in der Zitierung

\bibliography{_literatur}

\usepackage[]{acronym}																% für Abkürzungen

\usepackage{hyperref}																	% for hyperlink references
\usepackage[ngerman, nameinlink]{cleveref}						% for naming references

\addto\extrasngerman{\def\figureautorefname{Abb.}}		% changes figure reference text to "Abb."

\usepackage{float}							        							% for use of "H" specifier in floats
\usepackage[section]{placeins}          							% keep floats (images, tables, ..) in their place

% change symbols for unordered lists (itemize)
\renewcommand{\labelitemi}{$\bullet$}
\renewcommand{\labelitemii}{$\circ$}


\begin{document}

\mainmatter  % start of an individual contribution

% first the title is needed
\title{Erweiterungen des R-Baums für räumliche Datenbankanfragen}

\subtitle{Der R*-Baum}

% a short form should be given in case it is too long for the running head
% \titlerunning{Verschiebungsalgorithmen für Kartenbeschriftungen}

\author{Patrick Schulz \& Simon Hötten}

% the affiliations are given next; don't give your e-mail address
% unless you accept that it will be published
\institute{Seminar Geodatenbanken \\ Dozent: Prof. Dr.-Ing. Jan-Henrik Haunert \\ Institut für Geoinformatik und Fernerkundung\\ Universität Osnabrück \\ Sommersemester 2015}

%\toctitle{Titel vom Inhaltsverzeichnis}
%\tocauthor{Authors' Instructions}
\maketitle


% Zusammenfassung
%%%%%%%%%%%%%%%%%%%%%%%%%%%%%%%%%%%%%%%%%%%%%%%%%%%%%%%%%%%%%%%%%%%%
% \begin{abstract}

\keywords{Geodatenbanken, R*, Spatial Access}
% \end{abstract}


% Inhaltsverzeichnis
%%%%%%%%%%%%%%%%%%%%%%%%%%%%%%%%%%%%%%%%%%%%%%%%%%%%%%%%%%%%%%%%%%%%
% \tableofcontents
% \newpage


% Dokument
%%%%%%%%%%%%%%%%%%%%%%%%%%%%%%%%%%%%%%%%%%%%%%%%%%%%%%%%%%%%%%%%%%%%
\section{Motivation} % (fold)
\label{sec:intro}



% section intro (end)

\section{Prinzipien eines R-Baums} % (fold)
\label{sec:prinzipien_eines_r_baums}


% section prinzipien_eines_r_baums (end)

\section{Optimierungskriterien} % (fold)
\label{sec:optimierungskriterien}

	Bei dem herkömmlichen R-Baum wird, sowohl beim Hinzufügen neuer Elemente als auch beim Split, lediglich die Fläche der umschließenden Rechtecke minimiert \citep[vgl.][50-51]{Guttman:1984}. Einige der daraus resultierenden Probleme wurden bereits im vorherigen Abschnitt dargelegt.
	Im Folgenden werden weitere mögliche Optimierungen und ihre Wechselwirkungen aufgeführt.

	\subsection{Flächenausnutzung maximieren} % (fold)
	\label{sub:flaechenausnutzung}

	Die Fläche, welche von dem umschließenden Rechteck, aber nicht von den in ihm enthaltenen Rechtecken, überdeckt wird, soll minimiert werden. Es soll also möglichst wenig Platz \enquote{verschwendet} werden.

	% subsection flaechenausnutzung (end)

	\subsection{Überlappung minimieren} % (fold)
	\label{sub:ueberlappung_minimieren}

	Die Überlappung der umschließenden Rechtecke soll minimiert werden.
	
	% subsection ueberlappung_minimieren (end)

% section optimierungskriterien (end)

\section{Der R*-Baum} % (fold)
\label{sec:rstar_tree}



% section rstar_tree (end)

\section{Fazit} % (fold)
\label{sec:fazit}



% section fazit (end)



% Anhang
%%%%%%%%%%%%%%%%%%%%%%%%%%%%%%%%%%%%%%%%%%%%%%%%%%%%%%%%%%%%%%%%%%%%
\newpage
\begin{appendix}

	\section*{Anhang}
	\addcontentsline{toc}{section}{Anhang}

	\section*{Abkürzungsverzeichnis} % (fold)
	\label{sub:abbreviations}

		\begin{acronym}[length]
	    \acro{SAM}{Spatial access methods}
	    \acro{PAM}{Point access methods}
	    \acro{MBR}{Minimum bounding Rectangle}
	  \end{acronym}

	% section abbreviations (end)

	% Abbildungsverzeichnis
	% \listoffigures

	% Literaturverzeichnis
	%%%%%%%%%%%%%%%%%%%%%%%%%%%%%%%%%%%%%%%%%%%%%%%%%%%%%%%%%%%%%%%%%%%%
	\newpage
	\nocite{*}							% include all bibtex entries from bibliography, even if they are not citied in the document

	\printbibliography

\end{appendix}

\end{document}
